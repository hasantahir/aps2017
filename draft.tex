\documentclass[mathserif,16pt,xcolor=table]{beamer}

% Load Beamer Style Theme
% TAMU Based
\usepackage{tamu_beamer}
% \usepackage[skip=0pt]{caption}

% Specifiy the location of images to be used
\graphicspath{{figures/}}

% ------------------------------------------------------------
% ------------------------------------------------------------
% ------------------------------------------------------------
% ----------------TITLE PAGE----------------------------------
% ------------------------------------------------------------
% ------------------------------------------------------------
% ------------------------------------------------------------
% Document Title Page
\title{An Integral Equation Scheme for Plasma based Thin Sheets}
\author[Abbas and Nevels]{Hasan T. Abbas and Robert D. Nevels}
\institute{Electromagnetics \& Microwave Laboratory\\[\medskipamount] \pgfuseimage{tamuecenbig}}
\date[July 12, 2017]{2017 IEEE AP-S Symposium on Antennas and Propagation and USNC-URSI Radio Science Meeting\\
July 9-14, 2017\\
San Diego, California, USA}
\titlegraphic{\vspace*{-.55cm}\pgfuseimage{aps}}

% \date[July 11, 2017]{July 11, 2017}
% ------------------------------------------------------------
% ------------------------------------------------------------
% ------------------------------------------------------------
% ------------------------------------------------------------
% ------------------------------------------------------------
% ------------------------------------------------------------
% ------------------------------------------------------------
% ------------------------------------------------------------
% ------------------------------------------------------------
% ------------------------------------------------------------
% ------------------------------------------------------------
% ------------------------------------------------------------
% ------------------------------------------------------------
% ------------------------------------------------------------
\begin{document}

% Reduce space around equations and subequations
\preto\subequations{\ifhmode\unskip\fi}
\AtBeginEnvironment{subequations}{\ifhmode\unskip\fi}
\AtBeginEnvironment{equation}{\ifhmode\unskip\fi}

% Draw Boxes in the footer with pertinent info
\tikzstyle{block} = [rectangle, draw, rounded corners, shade, top color=white, text width=5em,
bottom color=blue!50!black!20, draw=blue!40!black!60, very thick, text centered, minimum height=4em]
\tikzstyle{line} = [draw, -latex']
\tikzstyle{cloud} = [draw, ellipse,top color=white, bottom color=red!20, node distance=2cm, minimum height=2em]

% Tick Style
\beamertemplateballitem
% \beamertemplatetransparentcoveredhigh

\frame{\titlepage}


% Add TAMU logo on each slide in the north-east side
% Shifted to be right at the edge
%
% NOTE: Will have to compile twice
%
\addtobeamertemplate{frametitle}{}{%
\begin{tikzpicture}[remember picture,overlay]
  \node[anchor=north east,yshift=7pt,xshift=2pt] at (current page.north east) {\includegraphics[height=.7cm]{ecen}};
\end{tikzpicture}}
% ------------------------------------------------------------
% ------------------------------------------------------------
% ------------------------------------------------------------
% ------------------------------------------------------------
% ------------------------------------------------------------
% ------------------------------------------------------------
% ------------------------------------------------------------
% ------------------------------------------------------------
% ------------------------------------------------------------
% ------------------------------------------------------------
% ------------------------------------------------------------
% ------------------------------------------------------------
% ------------------------------------------------------------
% ------------------------------------------------------------
\section{Outline}
\begin{frame}
  \frametitle{Outline}
  \begin{outline}[itemize]
    \1 Motivation and Objective
    \1 Background
    \1 Theory and Methods
      \2 Subwavelength phenomena - Dispersion relations
      \2 Existence of plasmonic behavior - Sommerfeld Integral analysis
      \2 Surface Integral equation scheme
    \1 Results
    \1 Conclusions
  \end{outline}
\end{frame}
% ------------------------------------------------------------
% ------------------------------------------------------------
% ------------------------------------------------------------
% ------------------------------------------------------------
% ------------------------------------------------------------
% ------------------------------------------------------------
% ------------------------------------------------------------
% ------------------------------------------------------------
% ------------------------------------------------------------
% ------------------------------------------------------------
% ------------------------------------------------------------
% ------------------------------------------------------------
% ------------------------------------------------------------
\section{Motivation and Objective}
% ------------------------------------------------------------
% ------------------------------------------------------------
\begin{frame}
  \frametitle{Motivation and Objective}
  % ----------------------------------------------------------
  % Talk about the significance of this research
  \begin{columns} % align columns
    \begin{column}{.5\textwidth} \vspace*{-1cm}
      \begin{outline}[itemize]
        \1 Plasmonics: subwavelength localization of electromagnetic (EM) fields
        \1 Bridging the THz gap
      \end{outline}
    \end{column}
    %
    \begin{column}{.5\textwidth}
      % Use this to preserve fonts from Inkspace
      \begin{figure}
        \centering \hspace*{-0.75cm}
        \fontsize{6}{7}\selectfont
        \def\svgwidth{1.1\linewidth}
        \input{figures/scale.pdf_tex}
        \label{fig:spp_2deg}
        \caption{Communication Technologies at various frequencies}
      \end{figure}
    \end{column}%
  \end{columns}
\end{frame}
% ------------------------------------------------------------
% ------------------------------------------------------------
% ------------------------------------------------------------
% ------------------------------------------------------------
% ------------------------------------------------------------
% ------------------------------------------------------------
% ------------------------------------------------------------
% ------------------------------------------------------------
% ------------------------------------------------------------
% \begin{frame}
%   \frametitle{Overview}
%   \framesubtitle{Terahertz 2DES}
%   % ----------------------------------------------------------
%   % Talk about the significance of this research
%   \begin{outline}[itemize]
%     \1 Graphene
%     \2 Grown separately, transferred to substrate
%     \2 Currently not integrable to current electronics technology
%     \2 Superior electronic properties
%     \1 Semiconductor heterostructures
%     \2 Conventional epitaxial semiconductor device fabrication techniques
%     \2 On-chip integration with silicon electronics
%   \end{outline}
% \end{frame}
%
% ------------------------------------------------------------
% ------------------------------------------------------------
% ------------------------------------------------------------
% ------------------------------------------------------------
% ------------------------------------------------------------
% ------------------------------------------------------------
% ------------------------------------------------------------
% ------------------------------------------------------------
% ------------------------------------------------------------
% ------------------------------------------------------------
% ------------------------------------------------------------
% ------------------------------------------------------------
% ------------------------------------------------------------
% ------------------------------------------------------------
% ------------------------------------------------------------
\begin{frame}
  \frametitle{Background}
  \framesubtitle{Two-dimensional Electron Gas (2DEG)}

  \begin{columns} % align columns
    \begin{column}{.5\textwidth}
      \begin{minipage}[T][.1\textheight][c]{\linewidth}
        \begin{outline}[itemize]
          \1 Semiconductor Heterostructure in high electron mobility transistor (HEMT)
          \1 High concentration of free electrons ($\sim \num{1e11}-\SI{1e14}{\cm^{-2}}$)
          \1 Very high Mobility ($\sim \num{1e3}-\SI{1e6}{\cm^{2}/V/s}$)
          \1 Formation of Quantum Well
          \2 Two-dimensional confinement of electrons
        \end{outline}
      \end{minipage}
      %
    \end{column}
    %
    \begin{column}{.5\textwidth}
      % Use this to preserve fonts from Inkspace
      \begin{figure}
        \centering
        \fontsize{6}{7}\selectfont
        \def\svgwidth{.8\linewidth}
        \input{figures/hemt2.pdf_tex}
        \caption{Typical GaAs/AlGaAs HEMT}
      \end{figure}
      \begin{figure}
        \centering \vspace*{-1cm}
        \fontsize{6}{7}\selectfont
        \def\svgwidth{.8\linewidth}
        \input{figures/2deg_bandgap.pdf_tex}
        \caption{Band diagram of a GaAs/AlGaAs heterostructure}
      \end{figure}
      \end{column}%
    \end{columns}
  \end{frame}
  % ------------------------------------------------------------
  % ------------------------------------------------------------
  % ------------------------------------------------------------
  % ------------------------------------------------------------
  % ------------------------------------------------------------
  % ------------------------------------------------------------
  % ------------------------------------------------------------
  % ------------------------------------------------------------
  \begin{frame}
    \frametitle{Background}
    \framesubtitle{2DEG (contd.)}

    \begin{columns} % align columns
      \begin{column}{.5\textwidth}
        \begin{minipage}[T][.1\textheight][c]{\linewidth}
          \begin{outline}[itemize]
            \1 Plasma waves in 2DEG
            \1 Dyakonov-Shur instability
            \2 Voltage bias at source and drain terminals
            \2 Plasma resonance
            \2 THz emission
            \1 Electronic Flute
            \2 Tunable resonance with gate voltage
            \1 Slow wave nature
              \2 Subwavelength propagation
          \end{outline}
        \end{minipage}
        %
      \end{column}
      %
      \begin{column}{.5\textwidth}
        % Use this to preserve fonts from Inkspace
        \vspace*{-1cm}
        \begin{figure}
          \hspace*{-.55cm}
          \fontsize{6}{7}\selectfont
          \def\svgwidth{1.1\linewidth}
          \input{figures/flute_2deg2.pdf_tex}
          % \caption{Typical GaAs/AlGaAs HEMT}
        \end{figure}
        \centering
        \begin{figure}
          \subfloat{\includegraphics[width= 2.5in]{com_sim.tikz}
          \label{fig:sim}}
        \end{figure}
        \end{column}%
      \end{columns}
    \end{frame}
% ------------------------------------------------------------
% ------------------------------------------------------------
% ------------------------------------------------------------
% ------------------------------------------------------------
% ------------------------------------------------------------
% ------------------------------------------------------------
% -------------------THEORY-----------------------------------
% ------------------------------------------------------------
% ------------------------------------------------------------
% ------------------------------------------------------------
% ------------------------------------------------------------
% ------------------------------------------------------------
% ------------------------------------------------------------
\begin{frame}
  \frametitle{Theory and Methods}
  \framesubtitle{2DEG Circuit model}
  \begin{columns} % align columns
    \begin{column}{.5\textwidth}
      \begin{minipage}[T][.1\textheight][c]{\linewidth}
        \begin{outline}[itemize]
          \1 Drude-Lorentz Surface Conductivity
        \end{outline}
        %
        \begin{equation} \nonumber
          \sigma_s = \frac{N_s e^2}{m^{\ast}}\frac{\tau}{1 + \j \tau \O}
          \label{eq:Y2deg}
        \end{equation}
        \begin{itemize}
          \item[] {\makebox[.3cm][l]{$N_s$} - Surface charge density}
          \item[] {\makebox[.3cm][l]{$\tau$} - Scattering time}
          \item[] {\makebox[.3cm][l]{$m^{\ast}$} - Effective electron mass}
        \end{itemize}
        \begin{equation} \nonumber
          k^{\mathrm{TM}}_{\mathrm P} = \frac{\O}{c} \sqrt{1 - \left(\frac{2}{\eta_0 \sigma_s}\right)^2}
          \label{eq:TM_pole}%
        \end{equation}
      \end{minipage}
    \end{column}
    %
    \begin{column}{.5\textwidth}
      % Use this to preserve fonts from Inkspace
      \begin{figure} \vspace*{-1.5cm}
        \includegraphics[height = 1.75in]{cond_2deg_gan.tikz}
        \label{fig:cond_2deg}
      \end{figure}
      %
      \begin{figure} \vspace*{-1.25cm}
        \subfloat{\includegraphics[height = 1.5in]{disp_gaas_sheet_77.tikz}
        \label{fig:disp_2deg_77}}
      \end{figure}
      \end{column}%
    \end{columns}
  \end{frame}
% ------------------------------------------------------------
% ------------------------------------------------------------
% ------------------------------------------------------------
% ------------------------------------------------------------
% ------------------------------------------------------------
% ------------------------------------------------------------
% ------------------------------------------------------------
% \begin{frame}
%   \frametitle{Theory and Methods}
%   \framesubtitle{Dispersion Relation for a 2D Sheet}
%   \begin{columns}[T] % align columns
%     \begin{column}{.5\textwidth}
%       \begin{outline}[itemize]
%         \1 Conductive Sheet in freespace
%         \1 TM mode surface wave
%       \end{outline}
%       \begin{equation} \nonumber
%         k^{\mathrm{TM}}_{\mathrm P} = \frac{\O}{c} \sqrt{1 - \left(\frac{2}{\eta_0 \sigma_s}\right)^2}
%         \label{eq:TM_pole}%
%       \end{equation}
%       \begin{outline}[itemize]
%         \1 Below plasma frequency
%       \end{outline}
%       \begin{equation} \nonumber
%         \Im \sigma_s < 0
%         \label{eq:2deg_surface_wave}%
%         \end{equation}%
%         \begin{outline}[itemize]
%           \1 At low temperature
%           \end{outline}%
%           \begin{equation} \nonumber
%             \Im |\sigma_s| \gg \Re |\sigma_s|
%             \label{eq:2deg_low_loss}
%           \end{equation}
%         \end{column}
%         \begin{column}[T]{.45\textwidth}
%           \vspace*{-2cm}
%           \centering
%           \begin{figure}
%             \subfloat{\includegraphics[height = 1.5in]{disp_gaas_sheet_77.tikz}
%             \label{fig:disp_2deg_77}}
%
%             \subfloat{\hspace*{-.1cm} \includegraphics[height = 1.5in]{disp_gaas_sheet_rt.tikz}
%             \label{fig:disp_2deg_rt}}
%           \end{figure}
%         \end{column}
%       \end{columns}
% \end{frame}
% ------------------------------------------------------------
% ------------------------------------------------------------
% ------------------------------------------------------------
% ------------------------------------------------------------
% ------------------------------------------------------------
% ------------------------------------------------------------
% ------------------------------------------------------------
% ------------------------------------------------------------
% ------------------------------------------------------------
% ------------------------------------------------------------
% ------------------------------------------------------------
% ------------------------------------------------------------
% ------------------------------------------------------------
% ------------------------------------------------------------
% \begin{frame}
%   \frametitle{Theory and Methods}
%   \framesubtitle{Field Computation of layered media}
%   \begin{columns}[T] % align columns
%     \begin{column}{.65\textwidth}
%       \begin{outline}[itemize]
%         \1 Methods using discretization computationally expensive
%         \2 Finite element method (FEM)
%         \2 Finite Difference Time Domain (FDTD)
%         \1 Integral equation (IE) approach most suitable
%         \2 Formulation of dyadic Green function (DGF)
%       \end{outline}
%       \begin{subequations} \nonumber
%         \begin{align}
%           \v E ={}& \int\limits_{r'} \tnsr{\v{G}}^\mathrm{EJ}(\v r | \v{r'}) \v{J}(\v{r'}) \diff{r'},
%           \label{eq:dyadic_E}\\
%           \v H ={}& \int\limits_{r'} \tnsr{\v{G}}^\mathrm{HJ}(\v r | \v{r'}) \v{J}(\v{r'}) \diff{r'}.
%           \label{eq:dyadic_H}
%         \end{align}
%         \label{eq:dyadic}%
%       \end{subequations}
%     \end{column}
%     \begin{column}[T]{.5\textwidth}
%       \begin{figure}
%         \centering \vspace*{-1.5cm} \hspace*{-1cm}
%         \fontsize{5}{6}\selectfont
%         \def\svgwidth{.4\linewidth}
%         \input{figures/flowchart.pdf_tex}
%         % \caption{Typical GaAs/AlGaAs HEMT}
%       \end{figure}
%     \end{column}
%   \end{columns}
% \end{frame}
% ------------------------------------------------------------
% ------------------------------------------------------------
% ------------------------------------------------------------
% ------------------------------------------------------------
% ------------------------------------------------------------
% ------------------------------------------------------------
% ------------------------------------------------------------
\begin{frame}
  \frametitle{Theory and Methods}
  \framesubtitle{Field Computation - Thin sheet}
  \begin{columns}[T] % align columns
    \begin{column}{.65\textwidth}
      \begin{outline}[itemize]
        \1 Thin conductive sheet in free-space
      \end{outline}
      \begin{equation} \nonumber
        Z^{\downarrow}(z_0^+) = \frac{Z_0}{ 1 + \sigma_s \, Z_0}
      \end{equation}
      %
      \begin{equation} \nonumber
        \Gamma^{\downarrow, \mathrm{TE}} = \frac{k_{z1} - \O \u_1 \sigma_s }{k_{z1} + \O \u_1 \sigma_s}
      \end{equation}
      %
      \begin{equation} \nonumber
        \Gamma^{\downarrow, \mathrm{TM}} = \frac{ \O \E_1 -\sigma_s k_{z1} }{\O \E_1 + \sigma_s k_{z1}}
      \end{equation}
      %
      \begin{equation} \nonumber
        E_z \approx \frac{\j \u}{2} \cos \phi \, \mathcal{S}_1 \left \{ \frac{\Gamma^{\downarrow, \mathrm{TM}} - \Gamma^{\downarrow, \mathrm{TE}}}{ k_{\p}} \right \}
        \label{eq:somm_Gzx}
      \end{equation}
      %
      \begin{equation} \nonumber
          \mathcal{S}_1\left \{\ti{F}\right \} \equiv \frac{1}{2\pi}\int\limits_0^{\inf} J_1(k_{\p}\p)  \ti{F}(k_{\p}) k_{\p} \diff{k_{\p}}.
        \label{eq:generalized_SI}%
      \end{equation}
    \end{column}
    \begin{column}[T]{.5\textwidth}
      \begin{figure}
        \centering \vspace*{-1.5cm} \hspace*{-1cm}
        \fontsize{6}{7}\selectfont
        \def\svgwidth{.5\linewidth}
        \input{figures/2deg_sheet.pdf_tex}
        % \caption{Typical GaAs/AlGaAs HEMT}
      \end{figure}
    \end{column}
  \end{columns}
\end{frame}
% ------------------------------------------------------------
% ------------------------------------------------------------
% ------------------------------------------------------------
% ------------------------------------------------------------
% ------------------------------------------------------------
% ------------------------------------------------------------
% ------------------------------------------------------------
\begin{frame}
  \frametitle{Theory and Methods}
  \framesubtitle{Computed Fields  - Thin sheet}
  \begin{figure} \centering
    \subfloat {\includegraphics[height = 2in]{Gzx_gan_sheet_56_rt.tikz}
    \label{fig:disp_2deg_77}}
    \hfil
    \subfloat {\includegraphics[height = 2in]{Gzx_gan_sheet_56_3k.tikz}
    \label{fig:disp_2deg_rt}}
    \caption{$G^{\mathrm A}_{zx}$ computed for a GaN/AlGaN based 2DEG sheet suspended in freespace at  $\SI[round-precision=2]{5.6}{\THz}$. The surface conductivity of the sheet is (a) $\sigma_s = \num[round-precision=2]{7.6e-5} - \num[round-precision=3]{j 2.98e-3} \SI{}{\siemens}$ at room temperature ($\SI{295}{\K}$), and (b) $\sigma_s = \num[round-precision=2]{7.6e-8} - \num[round-precision=3]{j 2.98e-3} \SI{}{\siemens}$ at $\SI{3}{\K}$ }
  \end{figure}
\end{frame}
% ------------------------------------------------------------
% ------------------------------------------------------------
% ------------------------------------------------------------
% ------------------------------------------------------------
% ------------------------------------------------------------
% ------------------------------------------------------------
% ------------------------------------------------------------
% ------------------------------------------------------------
% ------------------------------------------------------------
% ------------------------------------------------------------
% ------------------------------------------------------------
% ------------------------------------------------------------
% ------------------------------------------------------------
\begin{frame}
  \frametitle{Theory and Methods}
  \framesubtitle{Thin Sheet Simulation}
  \begin{columns}[T] % align columns
    \begin{column}{.5\textwidth}
      \begin{itemize}
        \item{Richmond's Volume Integral formulation}
      \end{itemize}
      \begin{equation} \nonumber
        \begin{split}
          \v A &= \frac{\u}{4 \pi} \int\limits_{V} \v J_v(\v r')  \frac{ \e^{-\j k_1 |\v r - \v r'|}}{|\v r - \v r'|} \diff{v'} \\
          \v E_1^{scat} &= -\frac{j \O}{k_1^2} \left( k_1^2 + \del \del \cdot \right) \v A \\
          \v J_v &= \frac{-\j k_1}{Z_0}(\E_2 - 1)\v E_2
        \end{split}
        \label{eq:E1sc}
      \end{equation}
      \begin{itemize}
        \item{Surface current $J_s$ approximated from $J_v$}
      \end{itemize}
    \end{column}
    \begin{column}[T]{.5\textwidth}
      \begin{itemize}
        \item{Senior's Impedance Boundary Condition}
      \end{itemize}
      \begin{equation} \nonumber
        \v E_{tan} = \eta Z_0 \hat{\v n} \times \v H
        \label{eq:eps}
      \end{equation}
      \begin{equation} \nonumber
        \begin{split}
          E^i &= \eta Z_0 J_s(x') \\
          & + \frac{\O \u}{4}  \int\limits_{l} J_s(x')  H_0^{(2)}(k_2 |x - x'|) \diff{x'}
        \end{split}
        \label{eq:eps}
      \end{equation}
      % Use this to preserve fonts from Inkspace
      \end{column}%
    \end{columns}
  \end{frame}
% ------------------------------------------------------------
% ------------------------------------------------------------
% ------------------------------------------------------------
% ------------------------------------------------------------
% ------------------------------------------------------------
% ------------------------------------------------------------
% ------------------------------------------------------------
% ------------------------------------------------------------
% ------------------------------------------------------------
% ------------------------------------------------------------
% ------------------------------------------------------------
% ------------------------------------------------------------
% ------------------------------------------------------------
% ------------------------------------------------------------
\begin{frame}
  \frametitle{Theory and Methods}
  \framesubtitle{Proposed Surface Integral Equation (SIE) scheme}

  \begin{itemize}
    \item{Surface Equivalence Theorem}
  \end{itemize}
  \begin{figure}
    \centering
    \def\svgwidth{1\linewidth}
    \input{figures/equivalence.pdf_tex}
    \caption{(a). Actual and its equivalent models for the (b) external and, (c) Internal region }
  \end{figure}
\end{frame}
% ------------------------------------------------------------
% ------------------------------------------------------------
% ------------------------------------------------------------
% ------------------------------------------------------------
% ------------------------------------------------------------
% ------------------------------------------------------------
% ------------------------------------------------------------
% \begin{frame}
%   \frametitle{Proposed Scheme}
%   \framesubtitle{Surface Integral Equation}
%   \vspace*{-.4cm}
%   \begin{itemize}
%     \item{Exterior Region}
%   \end{itemize}
%   \begin{equation} \nonumber
%     \begin{split}
%       \v E_1 &= \v E_i + \v E_1^{scat} \\
%       &=  -\frac{\O}{4 k_1^2} \left( k_1^2 + \del \del \cdot \right) \int\limits_{C} \v J_s(\v p') H_0^{(2)}(k_1 |\v \p - \v \p'|) \diff{l'} \\
%       &- \frac{1}{4 \E \j} \del \x \int\limits_{l} \v M_s(\v \p') H_0^{(2)}(k_1 |\v \p - \v \p'|) \diff{l'} + \v E_i
%     \end{split}
%     \label{eq:E1sc}
%   \end{equation}
%   \begin{equation} \nonumber
%     \begin{split}
%       \v H_1 &= \v H_i + \v H_1^{scat} \\
%       &= \frac{1}{4 \j} \del \x \int\limits_{l} \v J_s(\v \p') H_0^{(2)}(k_1 |\v \p - \v \p'|) \diff{l'} \\
%       &-\frac{\O}{4 k_1^2} \left( k_1^2 + \del \del \cdot \right) \int\limits_{l} \v M_s(\v \p') H_0^{(2)}(k_1 |\v \p - \v \p'|) \diff{l'} + \v H_i
%     \end{split}
%     \label{eq:H1sc}
%   \end{equation}
% \end{frame}
% ------------------------------------------------------------
% ------------------------------------------------------------
% ------------------------------------------------------------
% ------------------------------------------------------------
% ------------------------------------------------------------
% ------------------------------------------------------------
% ------------------------------------------------------------
% \begin{frame}
%   \frametitle{Proposed Scheme}
%   \framesubtitle{Surface Integral Equation}
%   \vspace*{-.4cm}
%   \begin{itemize}
%     \item{Interior Region}
%   \end{itemize}
%   \begin{equation} \nonumber
%     \begin{split}
%       \v E_2 &= \v E_2^{scat} \\
%       &=  -\frac{\O}{4 k_2^2} \left( k_2^2 + \del \del \cdot \right) \int\limits_{C} \left(-\v J_s(\v p')\right) H_0^{(2)}(k_2 |\v \p - \v \p'|) \diff{l'} \\
%       &- \frac{1}{4 \j} \del \x \int\limits_{l} \left(-\v M_s(\v \p')\right) H_0^{(2)}(k_2 |\v \p - \v \p'|) \diff{l'}
%     \end{split}
%     \label{eq:E1sc}
%   \end{equation}
%   \begin{equation} \nonumber
%     \begin{split}
%       \v H_2 &= \v H_1^{scat} \\
%       &= \frac{1}{4 \j} \del \x \int\limits_{l} \left(-\v J_s(\v \p')\right) H_0^{(2)}(k_2 |\v \p - \v \p'|) \diff{l'} \\
%       &-\frac{\O}{4 k_2^2} \left( k_2^2 + \del \del \cdot \right) \int\limits_{l} \left(-\v M_s(\v \p')\right) H_0^{(2)}(k_2 |\v \p - \v \p'|) \diff{l'}
%     \end{split}
%     \label{eq:H1sc}
%   \end{equation}
% \end{frame}
% ------------------------------------------------------------
% ------------------------------------------------------------
% ------------------------------------------------------------
% ------------------------------------------------------------
% ------------------------------------------------------------
% ------------------------------------------------------------
% ------------------------------------------------------------
\begin{frame}
  \frametitle{Theory and Methods}
  \framesubtitle{$TM_z$ SIE for Thin Flat Sheet}
  \begin{equation} \nonumber
    \hat{\v n} \x (\v E_1 - \v E_2) = \v 0
    \label{eq:E1bc}
  \end{equation}
  \begin{align} \nonumber
    E_i ={}& \frac{\O}{4} \int\limits_L J_z(x') \left[ H_0^{(2)}(k_1 | x -  x'|) + H_0^{(2)}(k_2 |x - x'|)\right] \diff{x'}  \notag
    \label{eq:scalarE}
  \end{align}
  \begin{equation} \nonumber
    \hat{\v n} \x (\v H_1 - \v H_2) = \v 0
    \label{eq:E1bc}
  \end{equation}
  \begin{align} \nonumber
    H_i^{tan} &= \frac{-j \O}{2} \int\limits_L M_x(x') \left[ \E_1 H_0^{(2)}(k_1 |x - x'|) + \E_1 H_2^{(2)}(k_1 |x - x'|) \right. \notag\\
    & \left. + \E_2 H_0^{(2)}(k_2 |x - x'|) + \E_2 H_2^{(2)}(k_2 |x - x'|)\right] \diff{x'} \notag
    \label{eq:scalarH}
  \end{align}
\end{frame}
% ------------------------------------------------------------
% ------------------------------------------------------------
% ------------------------------------------------------------
% ------------------------------------------------------------
% ------------------------------------------------------------
% ------------------------------------------------------------
% ------------------------------------------------------------
\begin{frame}
  \frametitle{Theory and Methods}
  \framesubtitle{Method of moments}
  \begin{itemize}
    \item {Integral equations to system of linear equations}
  \end{itemize}
  \[
  \begin{bmatrix}
    Z_{mn}   & 0 \\
    0        & Y_{mn}
  \end{bmatrix}
  \begin{bmatrix}
    J_n \\
    M_n
  \end{bmatrix}
  =
  \begin{bmatrix}
    E_m^i \\
    H_m^i
  \end{bmatrix}
  \]
  \begin{itemize}
    \item {Pulse basis functions and Point matching used}
    \item {Far-field}
  \end{itemize}
  \begin{equation} \nonumber
    RCS({\phi}) \simeq \int \limits_{0}^{L} \left[J_z(x')\eta_1 + M_x(x')\sin(\phi_i)\right] e^{j k_1 x' \cos(\phi_i)} \mathrm{d}x'
    \label{eq:far-field}
  \end{equation}
\end{frame}
% ------------------------------------------------------------
% ------------------------------------------------------------
% ------------------------------------------------------------
% ------------------------------------------------------------
% ------------------------------------------------------------
% ------------------------------------------------------------
% ------------------------------------------------------------
\section{Results and Applications}
\begin{frame}
  \frametitle{Results}
  \framesubtitle{Thin Sheet Simulation ($TM_z$)}
  \begin{columns}[T] % align columns
    \begin{column}{.5\textwidth}
      \begin{itemize}
        \item[-]{$TM_z$ polarization}
        \item[-]{Dielectric Rod of length 2.5 $\lambda$}
        \item[-]{$\E = 4$, $\u = 1$}
      \end{itemize}
      \begin{figure}
        \centering
        \includegraphics[height = 1.3in]{tm_plate1.tikz}
        \label{fig:tm_plate}
      \end{figure}
    \end{column}
    \begin{column}[T]{.5\textwidth}
      \begin{figure} \vspace*{-1cm}
        \includegraphics[height = 2in]{richmond_tm22.tikz}
        \label{fig:TM_rcs}
      \end{figure}
      \begin{itemize}
        \item[-]{Thickness of $.05 \lambda$ assumed in Volume Integral equation model}
      \end{itemize}
      \end{column}%
    \end{columns}
  \end{frame}
% ------------------------------------------------------------
% ------------------------------------------------------------
% ------------------------------------------------------------
% ------------------------------------------------------------
% ------------------------------------------------------------
% ------------------------------------------------------------
% ------------------------------------------------------------
\begin{frame}
  \frametitle{Results}
  \framesubtitle{Thin Sheet Simulation ($TE_z$)}
  \begin{columns}[T] % align columns
    \begin{column}{.5\textwidth}
      \begin{itemize}
        \item[-]{$TE_z$ polarization}
        \item[-]{Dielectric Rod of length 2.5 $\lambda$}
        \item[-]{$\E = 4$, $\u = 1$}
      \end{itemize}
      \begin{figure}
        \centering
        \includegraphics[height = 1.3in]{te_plate1.tikz}
        \label{fig:te_plate}
      \end{figure}
    \end{column}
    \begin{column}[T]{.5\textwidth}
      \begin{figure}
        \vspace*{-1cm}
        \includegraphics[height = 2in]{richmond_te11.tikz}
        \label{fig:TE_rcs}
      \end{figure}
      \begin{itemize}
        \item[-]{Thickness of $.05 \lambda$ assumed in Volume Integral equation model}
      \end{itemize}
      \end{column}%
    \end{columns}
  \end{frame}
% ------------------------------------------------------------
% ------------------------------------------------------------
% ------------------------------------------------------------
% ------------------------------------------------------------
% ------------------------------------------------------------
% ------------------------------------------------------------
% ------------------------------------------------------------
\begin{frame}
  \frametitle{Results}
  \framesubtitle{Thin Sheet Simulation ($TE_z$)}
  \begin{columns}[T] % align columns
    \begin{column}{.5\textwidth}
      \begin{itemize}
        \item[-]{$TE_z$ polarization}
        \item[-]{Dielectric Rod of length 2 $\lambda$}
        \item[-]{$\E = 4$, $\u = 1$}
      \end{itemize}
      \begin{figure}
        \centering
        \includegraphics[height = 1.3in]{te_plate1.tikz}
        \label{fig:te_plate}
      \end{figure}
    \end{column}
    \begin{column}[T]{.5\textwidth}
      \begin{figure}
        \vspace*{-1cm}
        \includegraphics[height = 2in]{farfield_22.tikz}
        % \caption{Radar Cross-section}
        \label{fig:TE_rcs}
      \end{figure}
      \begin{itemize}
        \item[-]{Thickness of $.628/k_1$ assumed in resistive model}
      \end{itemize}
      \end{column}%
    \end{columns}
  \end{frame}
  %
% ------------------------------------------------------------
% ------------------------------------------------------------
% ------------------------------------------------------------
% ------------------------------------------------------------
% ------------------------------------------------------------
% ------------------------------------------------------------
% ------------------------------------------------------------
% ------------------------------------------------------------
% ------------------------------------------------------------
% ------------------------------------------------------------
% ------------------------------------------------------------
% ------------------------------------------------------------
% ------------------------------------------------------------
% ------------------------------------------------------------
% ------------------------------------------------------------
% ------------------------------------------------------------
% ------------------------------------------------------------
% ------------------------------------------------------------
% ------------------------------------------------------------
% ------------------------------------------------------------
% ------------------------------------------------------------
\section{Conclusion}
\begin{frame}
  \frametitle{Summary}
  % \framesubtitle{Plasmonic Terahertz deTwo-dimensional plasmonic devices}
  \begin{itemize}
    \item Plasmonics in semiconductor transistor structures
    \item Realization of terahertz sources and sensors
    \item Scattering properties of infinitesimally thin plasma layers
    \item Temperature dependent performance limitations
  \end{itemize}
\end{frame}
% ------------------------------------------------------------
% ------------------------------------------------------------
% ------------------------------------------------------------
% ------------------------------------------------------------
% ------------------------------------------------------------
% ------------------------------------------------------------
% ------------------------------------------------------------
% \begin{frame}[plain,c]
%   \begin{center}
%     \Huge Thank you!
%   \end{center}
% \end{frame}
% ------------------------------------------------------------
% ------------------------------------------------------------
% ------------------------------------------------------------
% ------------------------------------------------------------
% ------------------------------------------------------------
% ------------------------------------------------------------
% ------------------------------------------------------------
% \begin{frame}[plain,c]
%   \begin{center}
%     \Huge Questions?
%   \end{center}
% \end{frame}
% ------------------------------------------------------------
% ------------------------------------------------------------
% ------------------------------------------------------------
% ------------------------------------------------------------
% ------------------------------------------------------------
% ------------------------------------------------------------
% ------------------------------------------------------------
\end{document}
